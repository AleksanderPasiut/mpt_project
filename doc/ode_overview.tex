\documentclass[aspectratio=169]{beamer}

\renewcommand\familydefault{\rmdefault}
\renewcommand\mathfamilydefault{\rmdefault}

\usepackage[utf8]{inputenc}
\usepackage[T1]{fontenc}
\usepackage{amsmath}
\usepackage{amsfonts}
\usepackage{amssymb}
\usepackage[normalem]{ulem}

\title{ODE in a nutshell}
\author[APasiut]{Aleksander Pasiut}

\usetheme{texsx}

\newcommand{\ee}[1]{	% equation short-command
\begin{equation}
#1
\end{equation}
}

\newcommand{\mat}[1]{	% matrix / vector short-command
\begin{bmatrix}
#1
\end{bmatrix}
}

\newcommand{\der}[0] { \partial } % derivatives
\newcommand{\dd}[2] { \frac{\partial #1}{\partial #2} }
\newcommand{\ddd}[2] { \frac{\partial^2 #1}{\partial #2^2} }
\newcommand{\ddx}[3] { \frac{\partial^2 #1}{\partial #2 \partial #3} }
\newcommand{\ddat}[3] { \frac{\partial #1}{\partial #2} \bigg\rvert_#3 }

\setbeamerfont{normal}{series=\sffamily,size={\fontsize{10}{10}}}

\setbeamercolor{itemize item}{fg=black}
\setbeamertemplate{itemize item}[square]

\setbeamercolor{itemize subitem}{fg=black}
\setbeamertemplate{itemize subitem}[triangle]

\let\tempone\itemize
\let\temptwo\enditemize
\renewenvironment{itemize}{\tempone\addtolength{\itemsep}{0.5\baselineskip}}{\temptwo}

\begin{document}
\usebeamerfont{normal}

\begin{frame}

    \setbeamerfont{title}{series=\bfseries,size={\fontsize{22}{22}}}
    \setbeamerfont{author}{series=\bfseries,size={\fontsize{14}{14}}}
    \vskip1cm
    \begin{beamercolorbox}[wd=\paperwidth]{title}
        \centering
        \usebeamerfont{title}
        ODE in a nutshell
    \end{beamercolorbox}
    \vskip2cm
    \begin{beamercolorbox}[wd=\paperwidth]{author}
        \centering
        \usebeamerfont{author}
        \insertauthor
    \end{beamercolorbox}

\end{frame}

\begin{frame}
\frametitle{Formal description}

    We consider ordinary differential equation of the form:
    \begin{equation}
        \frac{dx}{dt} (t) = F \big( x(t) \big)
    \end{equation}
    where $f : \mathbb{R}^n \to \mathbb{R}^n$ is an analytic function and we assume boundary condition:
    \begin{equation}
        x(0) = x_0.
    \end{equation}
    Solution to this equation is a function $x : \mathbb{R} \supset U \to \mathbb{R}^n$ where $U$ is some neighborhood of $0$.

\end{frame}

\begin{frame}
\frametitle{Simple case}

    Some differential equations can be solved with pen and paper. For example the equation:
    \begin{equation}
        \frac{dx}{dt} (t) = a x(t),
    \end{equation}
    for some real number $a$, with the boundary condition
    \begin{equation}
        x(0) = b
    \end{equation}
    where $b$ is some real number has the solution:
    \begin{equation}
        x(t) = b \exp( at ).
    \end{equation}

\end{frame}

\begin{frame}
\frametitle{Realistic case}

    \begin{itemize}
        \item<1-> In the real case we have to deal with ODEs where function $f$ is some nonlinear function and cannot be exactly solved.
        \item<2-> In such case numerical methods have to be used.
        \item<3-> One such method is Euler method - used mostly for teaching purposes.
        \item<4-> Another one is Runge-Kutta of order 4 - used very frequently in real use cases.
        \item<5-> CAPD implements Taylor-expansion based solver with interval arithmetics, that provides an exact bounds of the solution.
    \end{itemize}

\end{frame}

\end{document}
