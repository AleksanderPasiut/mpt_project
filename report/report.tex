\documentclass[12pt, a4paper]{article}

\title{CAPD-based web application}
\author{Aleksander Pasiut}

\begin{document}
\maketitle

\subsection*{PhD project overview}
The topic of my PhD project is "Computer assisted proofs in dynamical systems" and the goal is to prove the existence of a family of orbits oscillating to collision in the planar circular restricted three body problem. In other words, we provide a rigorous proof of the existence of a certain family of solutions of a particular ordinary differential equation. One of the tools that we use is a CAPD library. It is a C++ numerical library that features a rigorous solver for ordinary differential equations among other features.  

\subsection*{MPT project}
The project for the MPT course is partially related to the topic of my PhD. The idea is to develop a web interface for CAPD library. That way the user could access the library with an ordinary web browser, insert a differential equation and obtain the result. In order to realize this vision, the following tasks have to be performed: 
\begin{itemize}
\item implement general HTTP server; 
\item implement website with a graphical user interface; 
\item integrate CAPD library into the HTTP server. 
\end{itemize}

The tasks listed above form the mandatory scope of the project. The following additional features are proposed to make the application more reliable and user-friendly: 
\begin{itemize}
\item QR code for the ease of access;
\item computation time limitation (to prevent the server from hanging). 
\end{itemize}

\subsection*{Live demo}
It is considered essential for the project to succeed to perform a live demonstration of the application that would prove it to be operational and available for the external user to access. 

\subsection*{Sources}
http://capd.ii.uj.edu.pl/ 

\end{document}
